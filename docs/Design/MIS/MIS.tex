\documentclass[12pt, titlepage]{article}

\usepackage{amsmath, mathtools}

\usepackage[round]{natbib}
\usepackage{amsfonts}
\usepackage{amssymb}
\usepackage{graphicx}
\usepackage{colortbl}
\usepackage{xr}
\usepackage{hyperref}
\usepackage{longtable}
\usepackage{xfrac}
\usepackage{tabularx}
\usepackage{float}
\usepackage{siunitx}
\usepackage{booktabs}
\usepackage{multirow}
\usepackage[section]{placeins}
\usepackage{caption}
\usepackage{fullpage}

\hypersetup{
bookmarks=true,     % show bookmarks bar?
colorlinks=true,       % false: boxed links; true: colored links
linkcolor=red,          % color of internal links (change box color with linkbordercolor)
citecolor=blue,      % color of links to bibliography
filecolor=magenta,  % color of file links
urlcolor=cyan          % color of external links
}

\usepackage{array}

\externaldocument{../../SRS/SRS}

%% Comments

\usepackage{color}

\newif\ifcomments\commentstrue %displays comments
%\newif\ifcomments\commentsfalse %so that comments do not display

\ifcomments
\newcommand{\authornote}[3]{\textcolor{#1}{[#3 ---#2]}}
\newcommand{\todo}[1]{\textcolor{red}{[TODO: #1]}}
\else
\newcommand{\authornote}[3]{}
\newcommand{\todo}[1]{}
\fi

\newcommand{\wss}[1]{\authornote{blue}{SS}{#1}} 
\newcommand{\plt}[1]{\authornote{magenta}{TPLT}{#1}} %For explanation of the template
\newcommand{\an}[1]{\authornote{cyan}{Author}{#1}}

%% Common Parts

\newcommand{\progname}{Sayyara Automotive Matcher} % PUT YOUR PROGRAM NAME HERE
\newcommand{\authname}{Team 27, Kappastone
\\ Tevis Doe, doet
\\ Caitlin Bridel, bridelc
\\ Gilbert Cherrie, cherrieg
\\ Rachel Johnson, johnsr12
\\ Harkeerat Kanwal, kanwalh
\\ Himanshu Aggarwal, aggarwah} % AUTHOR NAMES                  

\usepackage{hyperref}
    \hypersetup{colorlinks=true, linkcolor=blue, citecolor=blue, filecolor=blue,
                urlcolor=blue, unicode=false}
    \urlstyle{same}
                                


\begin{document}

\title{Module Interface Specification for \progname{}}

\author{\authname}

\date{\today}

\maketitle

\pagenumbering{roman}

\section{Revision History}

\begin{tabularx}{\textwidth}{p{4cm}p{2cm}X}
\toprule {\bf Date} & {\bf Version} & {\bf Notes}\\
\midrule
January 18th, 2023 & 0.0 & Rev 0 \\
\bottomrule
\end{tabularx}

~\newpage

\section{Symbols, Abbreviations and Acronyms}

See SRS Documentation at \href{https://github.com/HKanwal/kapstone/blob/main/docs/SRS/SRS.pdf}{SRS for Sayyara Automotive Matcher} \\

\noindent\renewcommand{\arraystretch}{1.2}
\begin{tabular}{l l} 
  \toprule		
  \textbf{symbol} & \textbf{description}\\
  \midrule 
  AC & Anticipated Change\\
  DAG & Directed Acyclic Graph \\
  FR & Functional Requirement\\
  M & Module \\
  MG & Module Guide \\
  OS & Operating System \\
  PWA & Progressive Web Application \\
  \progname & PWA for maintenance appointment scheduling for vehicle owners\\
  SRS & Software Requirements Specification\\
  UC & Unlikely Change \\
  UI & User Interface \\
  \bottomrule
\end{tabular}\\

\newpage

\tableofcontents

\newpage

\pagenumbering{arabic}

\section{Introduction}

The following document details the Module Interface Specifications for
the Sayyara Automotive Matcher: An application designed to ease communcation between mechanics and potential clients.

Complementary documents include the System Requirement Specifications
and Module Guide.  The full documentation and implementation can be
found at \url{https://github.com/HKanwal/kapstone}. 

\section{Notation}

\wss{You should describe your notation.  You can use what is below as
  a starting point.}

The structure of the MIS for modules comes from \texit{Software Design, Automated Testing, and Maintenance: A Practical Approach} (Hoffman & Strooper, 1995),
with the addition that template modules have been adapted from
\texit{Fundamentals of Software Engineering} (Ghezzi et al., 2003).  The mathematical notation comes from Chapter 3 of
\texit{Software Design, Automated Testing, and Maintenance: A Practical Approach}.  For instance, the symbol := is used for a
multiple assignment statement and conditional rules follow the form $(c_1
\Rightarrow r_1 | c_2 \Rightarrow r_2 | ... | c_n \Rightarrow r_n )$.

The following table summarizes the primitive data types used by \progname. 

\begin{center}
\renewcommand{\arraystretch}{1.2}
\noindent 
\begin{tabular}{l l p{7.5cm}} 
\toprule 
\textbf{Data Type} & \textbf{Notation} & \textbf{Description}\\ 
\midrule
character & char & a single symbol or digit\\
integer & $\mathbb{Z}$ & a number without a fractional component in (-$\infty$, $\infty$) \\
natural number & $\mathbb{N}$ & a number without a fractional component in [1, $\infty$) \\
real & $\mathbb{R}$ & any number in (-$\infty$, $\infty$)\\
\bottomrule
\end{tabular} 
\end{center}

\noindent
The specification of \progname \ uses some derived data types: sequences, strings, and
tuples. Sequences are lists filled with elements of the same data type. Strings
are sequences of characters. Tuples contain a list of values, potentially of
different types. In addition, \progname \ uses functions, which
are defined by the data types of their inputs and outputs. Local functions are
described by giving their type signature followed by their specification.

\section{Module Decomposition}

The following table is taken directly from the Module Guide document for this project.

\begin{table}[h!]
\centering
\begin{tabular}{p{0.3\textwidth} p{0.6\textwidth}}
\toprule
\textbf{Level 1} & \textbf{Level 2}\\
\midrule

{Hardware-Hiding Module} & N/A \\
\midrule

\multirow{7}{0.3\textwidth}{Behaviour-Hiding Module}
& UI Module\\
& Authentication Module\\ 
& Dashboard Module\\
& Shop Creation Module\\ 
& Quote Request Module\\ 
& Quote Module\\
& Chat Module\\ 
& Account Information Module\\
& Work Order Module\\
& Service Module\\
& Appointment Module\\ 
\midrule

\multirow{3}{0.3\textwidth}{Software Decision Module}
& Available Appointments Module\\
& Update Appointments Module\\
& Appointment Slots Module\\
& Update Appointment Slots Module\\
\bottomrule

\end{tabular}
\caption{Module Hierarchy}
\label{TblMH}
\end{table}

\newpage
~\newpage

%%---UI MODULE---%%
\section{MIS of UI Module} \label{mUI}

\subsection{Module}

UI Module

\subsection{Uses}


\subsection{Syntax}

\subsubsection{Exported Constants}

None

\subsubsection{Exported Access Programs}

\begin{center}
\begin{tabular}{p{4cm} p{4cm} p{4cm} p{2cm}}
\hline
\textbf{Name} & \textbf{In} & \textbf{Out} & \textbf{Exceptions} \\
\hline
useThemePlease & string, string & - & - \\
\hline
\end{tabular}
\end{center}

\subsection{Semantics}

\subsubsection{State Variables}

primaryColor: string \\
secondaryColor: string

\subsubsection{Assumptions}

\subsubsection{Access Routine Semantics}

\noindent useThemePlease(primaryColor: string, secondaryColor: string):
\begin{itemize}
\item transition: $\text{primaryColor} = primaryColor \land \text{secondaryColor} = secondaryColor$
\item output: None 
\item exception: None
\end{itemize}

\subsubsection{Local Functions}

None

\newpage

%%---AUTHENTICATION MODULE---%%
\section{MIS of Authentication Module} \label{mAuth}

\subsection{Module}

Authentication Module

\subsection{Uses}


\subsection{Syntax}

\subsubsection{Exported Constants}

None

\subsubsection{Exported Access Programs}

\begin{center}
\begin{tabular}{p{6cm} p{3cm} p{1cm} p{5cm}}
\hline
\textbf{Name} & \textbf{In} & \textbf{Out} & \textbf{Exceptions} \\
\hline
pleaseCheckTheseCredentials & string, string & bool & SomethinAintRightException \\
pleaseRegisterMe & string, string & - & - \\
\hline
\end{tabular}
\end{center}

\subsection{Semantics}

\subsubsection{State Variables}
  
username: string \\
password: string

\subsubsection{Environment Variables}

\subsubsection{Assumptions}

\subsubsection{Access Routine Semantics}

\noindent pleaseCheckTheseCredentials(username, password):
\begin{itemize}
\item transition: None 
\item output: $\text{username} = username \land \text{password} = password$
\item exception: $(\text{username} != username \lor \text{password} != password) \implies \text{SomethinAintRightException}$
\end{itemize}

\noindent pleaseRegisterMe(username, password):
\begin{itemize}
\item transition: $\text{username} = username \land \text{password} = password$ 
\item output: None
\item exception: None
\end{itemize}

\subsubsection{Local Functions}

None

\newpage

%%---DASHBOARD MODULE---%%
\section{MIS of Dashboard Module} \label{mDashboard} \wss{Use labels for
  cross-referencing}

\wss{You can reference SRS labels, such as R\ref{R_Inputs}.}

\wss{It is also possible to use \LaTeX for hypperlinks to external documents.}

\subsection{Module}

Dashboard

\subsection{Uses}
UI

\subsection{Syntax}

\subsubsection{Exported Constants}
None
\subsubsection{Exported Access Programs}

\begin{center}
\begin{tabular}{p{4cm} p{3cm} p{3cm} p{2cm}}
\hline
\textbf{Name} & \textbf{In} & \textbf{Out} & \textbf{Exceptions} \\
\hline
getAccessiblePages & string &  &  \\
\hline
\end{tabular}
\end{center}

\subsection{Semantics}

\subsubsection{State Variables}

allPages: sequence of Pages \\
accessiblePages: sequence of Pages


\subsubsection{Environment Variables}


\subsubsection{Assumptions}

The arguments provided to access program will be of correct type.

\subsubsection{Access Routine Semantics}

\noindent getAccessiblePages(userType):
\begin{itemize}
\item transition: $(\forall i: \text{Page}| i \in \text{allPages} \land i.user = userType : \text{accessiblePages + i})$ 
\item output: None
\item exception: None
\end{itemize}

\subsubsection{Local Functions}

None

\newpage

%%---SHOP CREATION MODULE---%%
\section{MIS of Shop Creation Module} \label{mShopCreation}

\subsection{Module}

ShopCreation

\subsection{Uses}

UI

\subsection{Syntax}

\subsubsection{Exported Constants}

None

\subsubsection{Exported Access Programs}

\begin{center}
\begin{tabular}{p{2cm} p{4cm} p{4cm} p{2cm}}
\hline
\textbf{Name} & \textbf{In} & \textbf{Out} & \textbf{Exceptions} \\
\hline
newShop & sequence of strings &  & InvalidInfo \\
updateShop & integer, sequence of strings &  & InvalidID, InvalidInfo \\
\hline
\end{tabular}
\end{center}

\subsection{Semantics}

\subsubsection{State Variables}

\textit{shopInfo}: sequence of strings

\subsubsection{Access Routine Semantics}

\noindent newShop(input):
\begin{itemize}
\item transition: \textit{shopInfo} := \textit{input}
\item output: \textit{out} := \textit{self}
\item exception: ($\forall val \in |input| | invalidValue(val[i], i) \Rightarrow InvalidInfo$)
\end{itemize}

\noindent updateShop(shop, input):
\begin{itemize}
\item transition: \textit{shopInfo} := input
\item output: none
\item exceptions: ($invalidID(shop) \Rightarrow InvalidID$), ($\forall val \in |input| | invalidValue(val[i], i) \Rightarrow InvalidInfo$)
\end{itemize}

\subsubsection{Local Functions}

\noindent invalidValue(s, i): Verifies that \textit{s} is a valid parameter for a shop's data at position \textit{i}.

\noindent invalidID(shop): Verifies that the given shop ID corresponds with an existing shop in the API.

\newpage

%%---QUOTE REQUEST MODULE---%%
\section{MIS of Quote Request Module} \label{mQuoteRequest}

\subsection{Module}

QR Module

\subsection{Uses}

UI

\subsection{Syntax}

\subsubsection{Exported Constants}

\subsubsection{Exported Access Programs}

\begin{center}
\begin{tabular}{p{5cm} p{4cm} p{4cm} p{2cm}}
\hline
\textbf{Name} & \textbf{In} & \textbf{Out} & \textbf{Exceptions} \\
\hline
newQuoteRequest & sequence of strings &  & InvalidInfo \\
updateQuoteRequest & integer, sequence of strings &  & InvalidID, InvalidInfo \\
getQuoteRequest & integer & Quote Request & InvalidID \\
sendQuoteRequest & integer & & InvalidID \\
\hline
\end{tabular}
\end{center}

\subsection{Semantics}

\subsubsection{State Variables}

\textit{quoteRequestInfo}

\subsubsection{Access Routine Semantics}

\noindent newQuoteRequest(input):
\begin{itemize}
\item transition: \textit{quoteRequestInfo} := \texit{input}
\item output: \textit{out} := \textit{self}
\item exception: ($\forall val \in |input| | invalidValue(val[i], i) \Rightarrow InvalidInfo$)
\end{itemize}

\noindent updateQuoteRequest(quoteRequest, input):
\begin{itemize}
\item transition: \textit{quoteRequestInfo} := input
\item output: none
\item exceptions: ($invalidID(quoteRequest) \Rightarrow InvalidID$), ($\forall val \in |input| | invalidValue(val[i], i) \Rightarrow InvalidInfo$)
\end{itemize}

\noindent getQuoteRequest(quoteRequest):
\begin{itemize}
\item transition: none
\item output: \texit{out} := \texitit{quoteRequest}
\item exceptions: ($invalidID(quoteRequest) \Rightarrow InvalidID$)
\end{itemize}

\noindent sendQuoteRequest(quoteRequest):
\begin{itemize}
\item transition: quoteRequest $\rightarrow$ \hyperref[mQuote]{Quote Module}
\item output: none
\item exceptions: ($invalidID(quoteRequest) \Rightarrow InvalidID$)
\end{itemize}

\subsubsection{Local Functions}

\noindent invalidValue(s, i): Verifies that \textit{s} is a valid parameter for a shop's data at position \textit{i}.

\noindent invalidID(quoteRequest): Verifies that the given quote request ID corresponds with an existing quote request in the API.

\newpage

%%---QUOTE MODULE---%%
\section{MIS of Quote Module} \label{mQuote} \wss{Use labels for
  cross-referencing}

\subsection{Module}

Quote

\subsection{Uses}

UI

\subsection{Syntax}

\subsubsection{Exported Constants}

\subsubsection{Exported Access Programs}

\begin{center}
\begin{tabular}{p{5cm} p{4cm} p{4cm} p{2cm}}
\hline
\textbf{Name} & \textbf{In} & \textbf{Out} & \textbf{Exceptions} \\
\hline
newQuote & sequence of strings &  & InvalidInfo \\
updateQuote & integer, sequence of strings &  & InvalidID, InvalidInfo \\
getQuote & integer & Quote & InvalidID \\
sendQuote & integer & & InvalidID \\
\hline
\end{tabular}
\end{center}

\subsection{Semantics}

\subsubsection{State Variables}

\textit{quoteInfo}

\subsubsection{Access Routine Semantics}

\noindent newQuote(input):
\begin{itemize}
\item transition: \textit{quoteInfo} := \texit{input}
\item output: \textit{out} := \textit{self}
\item exception: ($\forall val \in |input| | invalidValue(val[i], i) \Rightarrow InvalidInfo$)
\end{itemize}

\noindent updateQuote(quote, input):
\begin{itemize}
\item transition: \textit{quoteInfo} := input
\item output: none
\item exceptions: ($invalidID(quote) \Rightarrow InvalidID$), ($\forall val \in |input| | invalidValue(val[i], i) \Rightarrow InvalidInfo$)
\end{itemize}

\noindent getQuote(quote):
\begin{itemize}
\item transition: none
\item output: \texit{out} := \texitit{quote}
\item exceptions: ($invalidID(quote) \Rightarrow InvalidID$)
\end{itemize}

\noindent sendQuote(quote):
\begin{itemize}
\item transition: quote $\rightarrow$ \hyperref[mQuoteRequest]{Quote Request Module}
\item output: none
\item exceptions: ($invalidID(quote) \Rightarrow InvalidID$)
\end{itemize}

\subsubsection{Local Functions}

\noindent invalidValue(s, i): Verifies that \textit{s} is a valid parameter for a shop's data at position \textit{i}.

\noindent invalidID(quote): Verifies that the given quote ID corresponds with an existing quote in the API.

\newpage

%%---CHAT MODULE---%%
\section{MIS of Chat Module} \label{mChat} \wss{Use labels for
  cross-referencing}

\wss{You can reference SRS labels, such as R\ref{R_Inputs}.}

\wss{It is also possible to use \LaTeX for hypperlinks to external documents.}

\subsection{Module}

Chat

\wss{Short name for the module}

\subsection{Uses}

UI

\subsection{Syntax}

\subsubsection{Exported Constants}

None

\subsubsection{Exported Access Programs}

\begin{center}
\begin{tabular}{p{2cm} p{4cm} p{4cm} p{2cm}}
\hline
\textbf{Name} & \textbf{In} & \textbf{Out} & \textbf{Exceptions} \\
\hline
new Chat & string & Chat & \\
SendMessage & string & & \\
GetMessages & & sequence of strings & \\
\hline
\end{tabular}
\end{center}

\subsection{Semantics}

\subsubsection{State Variables}

recipient: string

\subsubsection{Environment Variables}

\subsubsection{Assumptions}

\subsubsection{Access Routine Semantics}

\noindent new Chat(r):
\begin{itemize}
\item transition: \textit{recipient} := \textit{r}
\item output:\textit{out} := \textit{self}
\item exception: none
\end{itemize}

\noindent SendMessage(s):
\begin{itemize}
\item transition: send s to API
\item exception: none
\end{itemize}

\noindent GetMessages():
\begin{itemize}
\item transition: get messages from API
\item output: \textit{out} := sequence of strings
\item exception: none
\end{itemize}

\subsubsection{Local Functions}

None

\newpage

%%---ACCOUNT INFORMATION MODULE---%%
\section{MIS of Account Information Module} \label{mAccountInfo} \wss{Use labels for
  cross-referencing}

\wss{You can reference SRS labels, such as R\ref{R_Inputs}.}

\wss{It is also possible to use \LaTeX for hypperlinks to external documents.}

\subsection{Module}

AccountInformation

\subsection{Uses}

UI

\subsection{Syntax}

\subsubsection{Exported Constants}

None

\subsubsection{Exported Access Programs}

\begin{center}
\begin{tabular}{p{4cm} p{3cm} p{3cm} p{2cm}}
\hline
\textbf{Name} & \textbf{In} & \textbf{Out} & \textbf{Exceptions} \\
\hline
new AccountInformation & sequence of strings & AccountInformation & Invalid Information \\
getInfo & & sequence of strings & \\
edit & sequence of strings & & Invalid Information \\
delete & & & \\
\hline
\end{tabular}
\end{center}

\subsection{Semantics}

\subsubsection{State Variables}

info: sequence of strings

\subsubsection{Environment Variables}

\subsubsection{Assumptions}

\subsubsection{Access Routine Semantics}

\noindent new AccountInformation(i):
\begin{itemize}
\item transition: \textit{info} := \textit{i}
\item output: \textit{out} := \textit{self}
\item exception: (\forall val \in |i| | invalidValue(val[i], i) \Rightarrow InvalidInformation)
\end{itemize}

\hfill

\noindent getInfo():
\begin{itemize}
\item output: \textit{out} := \textit{info}
\item exception: None
\end{itemize}

\noindent edit(i):
\begin{itemize}
\item transition: \textit{info} := \textit{i}
\item exception: (\forall val \in |i| | invalidValue(val[i], i) \Rightarrow InvalidInformation)
\end{itemize}

\hfill

\noindent delete(i):
\begin{itemize}
\item transition:  \textit{info} := \textit{[]}
\item exception: None
\end{itemize}

\subsubsection{Local Functions}

invalidValue(s,i): check that s is a valid value for a string in position i

\newpage

%%---WORK ORDER MODULE---%%
\section{MIS of Work Order Module} \label{mWorkOrder}

\subsection{Module}

WorkOrder

\subsection{Uses}
UI

\subsection{Syntax}

\subsubsection{Exported Constants}
None

\subsubsection{Exported Access Programs}

\begin{center}
\begin{tabular}{p{4cm} p{5cm} p{2cm} p{2cm}}
\hline
\textbf{Name} & \textbf{In} & \textbf{Out} & \textbf{Exceptions} \\
\hline
updateWorkOrder & int, sequnece of strings &  & InvalidID, InvalidInfo \\
getWorkOrder & int & workOrder & InvalidID \\
\hline
\end{tabular}
\end{center}

\subsection{Semantics}

\subsubsection{State Variables}

workOrderInfo: sequence of strings

\subsubsection{Environment Variables}

None

\subsubsection{Assumptions}

None

\subsubsection{Access Routine Semantics}

\noindent updateWorkOrder(workOrder, input):
\begin{itemize}
\item transition: \textit{workOrderInfo} := \textit{input}
\item output: none
\item exception: ($invalidID(workOrder) \Rightarrow InvalidID$), ($\forall val \in |input| | invalidValue(val[i], i) \Rightarrow InvalidInfo$)
\end{itemize}

\noindent getWorkOrder(workOrder):
\begin{itemize}
\item transition: none
\item output: \texit{out} := \texitit{workOrder}
\item exceptions: ($invalidID(workOrder) \Rightarrow InvalidID$)
\end{itemize}


\subsubsection{Local Functions}

\noindent invalidValue(s, i): Verifies that \textit{s} is a valid parameter for a shop's data at position \textit{i}.

\noindent invalidID(workOrder): Verifies that the given work order ID corresponds with an existing work order in the API.

\newpage

%%---SERVICE MODULE---%%
\section{MIS of Service Module} \label{mService}

\subsection{Module}

Service

\subsection{Uses}

UI

\subsection{Syntax}

\subsubsection{Exported Constants}

None

\subsubsection{Exported Access Programs}

\begin{center}
\begin{tabular}{p{3cm} p{4cm} p{4cm} p{2cm}}
\hline
\textbf{Name} & \textbf{In} & \textbf{Out} & \textbf{Exceptions} \\
\hline
newService & sequence of strings &  & InvalidInfo \\
updateService & int, sequence of strings &  & InvalidInfo, InvalidID \\
getService & int & Service & InvalidID \\
\hline
\end{tabular}
\end{center}

\subsection{Semantics}

\subsubsection{State Variables}

serviceInfo: sequence of strings

\subsubsection{Environment Variables}


\subsubsection{Assumptions}
None

\subsubsection{Access Routine Semantics}

\noindent newService(input):
\begin{itemize}
\item transition: \textit{serviceInfo} := \texit{input}
\item output: \textit{out} := \textit{self}
\item exception: ($\forall val \in |input| | invalidValue(val[i], i) \Rightarrow InvalidInfo$)
\end{itemize}

\noindent updateService(service, input):
\begin{itemize}
\item transition: \textit{serviceInfo} := input
\item output: none
\item exceptions: ($invalidID(service) \Rightarrow InvalidID$), ($\forall val \in |input| | invalidValue(val[i], i) \Rightarrow InvalidInfo$)
\end{itemize}

\noindent getService(service):
\begin{itemize}
\item transition: none
\item output: \texit{out} := \texitit{service}
\item exceptions: ($invalidID(service) \Rightarrow InvalidID$)
\end{itemize}

\subsubsection{Local Functions}

\noindent invalidValue(s, i): Verifies that \textit{s} is a valid parameter for a shop's data at position \textit{i}.

\noindent invalidID(service): Verifies that the given service ID corresponds with an existing service in the API.

\newpage

%%---APPOINTMENT MODULE---%%
\section{MIS of Appointment Module} \label{mAppointment} \wss{Use labels for
  cross-referencing}

\wss{You can reference SRS labels, such as R\ref{R_Inputs}.}

\wss{It is also possible to use \LaTeX for hypperlinks to external documents.}

\subsection{Module}

Appointment

\subsection{Uses}

UI

\subsection{Syntax}

\subsubsection{Exported Constants}

None

\subsubsection{Exported Access Programs}

\begin{center}
\begin{tabular}{p{2cm} p{4cm} p{4cm} p{2cm}}
\hline
\textbf{Name} & \textbf{In} & \textbf{Out} & \textbf{Exceptions} \\
\hline
create & string, string & & \\
get & string & string & \\
edit & string, string & & \\
\hline
\end{tabular}
\end{center}

\subsection{Semantics}

\subsubsection{State Variables}

appointments: sequence of sequences of strings

\subsubsection{Environment Variables}

\subsubsection{Assumptions}

\subsubsection{Access Routine Semantics}

\noindent create(name, time):
\begin{itemize}
\item transition: \textit{appointments} := \textit{appointments + [name,time]}
\item exception: None
\end{itemize}

\noindent get(name):
\begin{itemize}
\item output: \textit{output} := \textit{appointments[i] where appointments[i][0] = name}
\item exception: None
\end{itemize}

\noindent create(name, time):
\begin{itemize}
\item transition: \textit{appointments} := \textit{appointments + [name,time]}
\item exception: None
\end{itemize}

\subsubsection{Local Functions}

None

\newpage

%%---AVAILABLE APPOINTMENTS MODULE---%%
\section{MIS of Available Appointments Module} \label{mAvailableAppointments}

\subsection{Module}
Available Appointments

\subsection{Uses}
None

\subsection{Syntax}

\subsubsection{Exported Constants}
None
\subsubsection{Exported Access Programs}
\begin{center}
\begin{tabular}{p{2cm} p{4cm} p{4cm} p{2cm}}
\hline
\textbf{Name} & \textbf{In} & \textbf{Out} & \textbf{Exceptions} \\
\hline
list & String, Date, Date, Integer & Dictionary & None \\
\hline
\end{tabular}
\end{center}

\subsection{Semantics}

\subsubsection{State Variables}
None

\subsubsection{Environment Variables}
$apt\_slots$: Database table containing Appointment Slots (set)

\subsubsection{Assumptions}
None

\subsubsection{Access Routine Semantics}
\noindent list(shop, from, to, duration):
\begin{itemize}
\item transition: None
\item output: 
out := \{$e: AppointmentSlot \; | \; e \in apt\_slots \; \land \; e.from \ge from \; \land \; e.to \le to \; \land \; e.shop = shop : get\_slots(e, duration)$\}
\item exception: None 
\end{itemize}


\subsubsection{Local Functions}
get\_slots : AppointmentSlot $\times$ Duration $\rightarrow$ AppointmentSlotSet \\
get\_slots(a, d) $\equiv$ $(s:AppointmentSlotSet \; | \; s[0] = a \; \land \; s[|s| \; - \; 1] = d : s)$

\newpage

%%---UPDATE APPOINTMENTS MODULE---%%
\section{MIS of Update Appointments Module} \label{mUpdateAppointments}

\subsection{Module}
Update Appointments

\subsection{Uses}
None

\subsection{Syntax}

\subsubsection{Exported Constants}
None

\subsubsection{Exported Access Programs}

\begin{center}
\begin{tabular}{p{2cm} p{4cm} p{4cm} p{2cm}}
\hline
\textbf{Name} & \textbf{In} & \textbf{Out} & \textbf{Exceptions} \\
\hline
update & String, Date & None & None \\
\hline
\end{tabular}
\end{center}

\subsection{Semantics}

\subsubsection{State Variables}
None

\subsubsection{Environment Variables}
$apt\_slots$: Database table containing Appointment Slots (set)

\subsubsection{Assumptions}
None

\subsubsection{Access Routine Semantics}
\noindent update(shop, date):
\begin{itemize}
\item transition: \\
$(\forall i:AppointmentSlot \; | \; i \in apt\_slots \; \land \; i.date = date \; \land \; i.shop = shop \; \land \; i.is\_cancelled : cancel(i.appointments))$
\item output: None
\item exception: None
\end{itemize}


\subsubsection{Local Functions}
cancel : set of Appointment $\rightarrow None$ \\
cancel(s) $\equiv$ $(\forall j:Appointment| j \in s : j.cancel())$

\newpage

%%---APPOINTMENT SLOTS MODULE---%%
\section{MIS of Appointment Slots Module} \label{mAppointmentSlots} 

\subsection{Module}
Appointment Slots

\subsection{Uses}
None

\subsection{Syntax}

\subsubsection{Exported Constants}
None

\subsubsection{Exported Access Programs}
\begin{center}
\begin{tabular}{p{2cm} p{4cm} p{4cm} p{2cm}}
\hline
\textbf{Name} & \textbf{In} & \textbf{Out} & \textbf{Exceptions} \\
\hline
generate & String, Date & None & None \\
\hline
\end{tabular}
\end{center}

\subsection{Semantics}

\subsubsection{State Variables}
None

\subsubsection{Environment Variables}
$apt\_slots$: Database table containing Appointment Slots (set)

\subsubsection{Assumptions}
None

\subsubsection{Access Routine Semantics}
\noindent generate(shop, date):
\begin{itemize}
\item transition: \\
$apt\_slots := \text{apt\_slots} \; \cup \; \text{AppointmentSlotSet}(\langle e: ShopAvailability |: generate\_slots(e) \rangle)$
\item output: None
\item exception: None
\end{itemize}

\subsubsection{Local Functions}
generate\_slots : ShopAvailability $\rightarrow$ set of AppointmentSlot \\
generate\_slots(s) $\equiv$ $\{i:\mathbb{N}| i \in [s.start..s.end, s.step] : \text{new AppointmentSlot}(s.shop, i, i+s.step) \}$

\newpage

%%---UPDATE APPOINTMENT SLOTS MODULE---%%
\section{MIS of Update Appointment Slots Module} \label{mUpdateAppointmentSlots}

\subsection{Module}
Update Appointment Slots

\subsection{Uses}
None

\subsection{Syntax}

\subsubsection{Exported Constants}
None

\subsubsection{Exported Access Programs}
\begin{center}
\begin{tabular}{p{2cm} p{4cm} p{4cm} p{2cm}}
\hline
\textbf{Name} & \textbf{In} & \textbf{Out} & \textbf{Exceptions} \\
\hline
update & String, Date & None & None \\
\hline
\end{tabular}
\end{center}

\subsection{Semantics}

\subsubsection{State Variables}
None

\subsubsection{Environment Variables}
$apt\_slots$: Database table containing Appointment Slots (set)

\subsubsection{Assumptions}
None

\subsubsection{Access Routine Semantics}
\noindent update(shop, date):
\begin{itemize}
\item transition: \\
$apt\_slots := (\text{AppointmentSlotSet}(\langle e: AppointmentSlot | e.date = date \land e.shop = shop \land check\_hours(e, shop.hours, date): e \rangle))$
\item output: None
\item exception: None
\end{itemize}

\subsubsection{Local Functions}
check\_hours : AppointmentSlot $\times$ set of ShopAvailability $\times$ Date $\rightarrow \mathbb{B}$ \\
check\_hours(e, hours, date) $\equiv$ $(\exists i:\mathbb{N}| i \in [0..|hours|-1] : e.start \ge hours[i].start  \land e.end \le hours[i].end)$

\newpage

\bibliographystyle {plainnat}
\bibliography {../../../refs/References}

\newpage

%\section{Appendix} \label{Appendix}

\wss{Extra information if required}

\end{document}